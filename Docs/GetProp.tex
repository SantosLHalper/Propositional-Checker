\documentclass{article}
%
%
\makeatletter
\@ifundefined{lhs2tex.lhs2tex.sty.read}%
  {\@namedef{lhs2tex.lhs2tex.sty.read}{}%
   \newcommand\SkipToFmtEnd{}%
   \newcommand\EndFmtInput{}%
   \long\def\SkipToFmtEnd#1\EndFmtInput{}%
  }\SkipToFmtEnd

\newcommand\ReadOnlyOnce[1]{\@ifundefined{#1}{\@namedef{#1}{}}\SkipToFmtEnd}
\usepackage{amstext}
\usepackage{amssymb}
\usepackage{stmaryrd}
\DeclareFontFamily{OT1}{cmtex}{}
\DeclareFontShape{OT1}{cmtex}{m}{n}
  {<5><6><7><8>cmtex8
   <9>cmtex9
   <10><10.95><12><14.4><17.28><20.74><24.88>cmtex10}{}
\DeclareFontShape{OT1}{cmtex}{m}{it}
  {<-> ssub * cmtt/m/it}{}
\newcommand{\texfamily}{\fontfamily{cmtex}\selectfont}
\DeclareFontShape{OT1}{cmtt}{bx}{n}
  {<5><6><7><8>cmtt8
   <9>cmbtt9
   <10><10.95><12><14.4><17.28><20.74><24.88>cmbtt10}{}
\DeclareFontShape{OT1}{cmtex}{bx}{n}
  {<-> ssub * cmtt/bx/n}{}
\newcommand{\tex}[1]{\text{\texfamily#1}}	% NEU

\newcommand{\Sp}{\hskip.33334em\relax}
\newcommand{\NB}{\textbf{NB}}
\newcommand{\Todo}[1]{$\langle$\textbf{To do:}~#1$\rangle$}

\EndFmtInput
\makeatother
%
%
%
%
%
%
% This package provides two environments suitable to take the place
% of hscode, called "plainhscode" and "arrayhscode". 
%
% The plain environment surrounds each code block by vertical space,
% and it uses \abovedisplayskip and \belowdisplayskip to get spacing
% similar to formulas. Note that if these dimensions are changed,
% the spacing around displayed math formulas changes as well.
% All code is indented using \leftskip.
%
% Changed 19.08.2004 to reflect changes in colorcode. Should work with
% CodeGroup.sty.
%
\ReadOnlyOnce{polycode.fmt}%
\makeatletter

\newcommand{\hsnewpar}[1]%
  {{\parskip=0pt\parindent=0pt\par\vskip #1\noindent}}

% can be used, for instance, to redefine the code size, by setting the
% command to \small or something alike
\newcommand{\hscodestyle}{}

% The command \sethscode can be used to switch the code formatting
% behaviour by mapping the hscode environment in the subst directive
% to a new LaTeX environment.

\newcommand{\sethscode}[1]%
  {\expandafter\let\expandafter\hscode\csname #1\endcsname
   \expandafter\let\expandafter\endhscode\csname end#1\endcsname}

% "compatibility" mode restores the non-polycode.fmt layout.

\newenvironment{compathscode}%
  {\par\noindent
   \advance\leftskip\mathindent
   \hscodestyle
   \let\\=\@normalcr
   \let\hspre\(\let\hspost\)%
   \pboxed}%
  {\endpboxed\)%
   \par\noindent
   \ignorespacesafterend}

\newcommand{\compaths}{\sethscode{compathscode}}

% "plain" mode is the proposed default.
% It should now work with \centering.
% This required some changes. The old version
% is still available for reference as oldplainhscode.

\newenvironment{plainhscode}%
  {\hsnewpar\abovedisplayskip
   \advance\leftskip\mathindent
   \hscodestyle
   \let\hspre\(\let\hspost\)%
   \pboxed}%
  {\endpboxed%
   \hsnewpar\belowdisplayskip
   \ignorespacesafterend}

\newenvironment{oldplainhscode}%
  {\hsnewpar\abovedisplayskip
   \advance\leftskip\mathindent
   \hscodestyle
   \let\\=\@normalcr
   \(\pboxed}%
  {\endpboxed\)%
   \hsnewpar\belowdisplayskip
   \ignorespacesafterend}

% Here, we make plainhscode the default environment.

\newcommand{\plainhs}{\sethscode{plainhscode}}
\newcommand{\oldplainhs}{\sethscode{oldplainhscode}}
\plainhs

% The arrayhscode is like plain, but makes use of polytable's
% parray environment which disallows page breaks in code blocks.

\newenvironment{arrayhscode}%
  {\hsnewpar\abovedisplayskip
   \advance\leftskip\mathindent
   \hscodestyle
   \let\\=\@normalcr
   \(\parray}%
  {\endparray\)%
   \hsnewpar\belowdisplayskip
   \ignorespacesafterend}

\newcommand{\arrayhs}{\sethscode{arrayhscode}}

% The mathhscode environment also makes use of polytable's parray 
% environment. It is supposed to be used only inside math mode 
% (I used it to typeset the type rules in my thesis).

\newenvironment{mathhscode}%
  {\parray}{\endparray}

\newcommand{\mathhs}{\sethscode{mathhscode}}

% texths is similar to mathhs, but works in text mode.

\newenvironment{texthscode}%
  {\(\parray}{\endparray\)}

\newcommand{\texths}{\sethscode{texthscode}}

% The framed environment places code in a framed box.

\def\codeframewidth{\arrayrulewidth}
\RequirePackage{calc}

\newenvironment{framedhscode}%
  {\parskip=\abovedisplayskip\par\noindent
   \hscodestyle
   \arrayrulewidth=\codeframewidth
   \tabular{@{}|p{\linewidth-2\arraycolsep-2\arrayrulewidth-2pt}|@{}}%
   \hline\framedhslinecorrect\\{-1.5ex}%
   \let\endoflinesave=\\
   \let\\=\@normalcr
   \(\pboxed}%
  {\endpboxed\)%
   \framedhslinecorrect\endoflinesave{.5ex}\hline
   \endtabular
   \parskip=\belowdisplayskip\par\noindent
   \ignorespacesafterend}

\newcommand{\framedhslinecorrect}[2]%
  {#1[#2]}

\newcommand{\framedhs}{\sethscode{framedhscode}}

% The inlinehscode environment is an experimental environment
% that can be used to typeset displayed code inline.

\newenvironment{inlinehscode}%
  {\(\def\column##1##2{}%
   \let\>\undefined\let\<\undefined\let\\\undefined
   \newcommand\>[1][]{}\newcommand\<[1][]{}\newcommand\\[1][]{}%
   \def\fromto##1##2##3{##3}%
   \def\nextline{}}{\) }%

\newcommand{\inlinehs}{\sethscode{inlinehscode}}

% The joincode environment is a separate environment that
% can be used to surround and thereby connect multiple code
% blocks.

\newenvironment{joincode}%
  {\let\orighscode=\hscode
   \let\origendhscode=\endhscode
   \def\endhscode{\def\hscode{\endgroup\def\@currenvir{hscode}\\}\begingroup}
   %\let\SaveRestoreHook=\empty
   %\let\ColumnHook=\empty
   %\let\resethooks=\empty
   \orighscode\def\hscode{\endgroup\def\@currenvir{hscode}}}%
  {\origendhscode
   \global\let\hscode=\orighscode
   \global\let\endhscode=\origendhscode}%

\makeatother
\EndFmtInput
%
\begin{document}
My first attempt at producing something remotely close to my Propositional
Checker website. A modified version of the RqDataUpload.hs source from the
Happstack Crash Course.

The advice given to me is that I need to have two pages generated. A
ServerPart Response is a page. One page will setup the form to collect a
proposition from the student and post it, and the next page will parse that
information and generate a Response page, which will display whether the
proposition is correct or incorrect, and possibly provide feedback.

Possibly use 'lookInput' with the RqData monad. It actually turned out that I
could just use 'look'.

I will need to process the string I get and then display the result.

\begin{tabbing}\tt
~\char123{}\char45{}\char35{}~LANGUAGE~OverloadedStrings~\char35{}\char45{}\char125{}\\
\tt ~import~Control\char46{}Monad~~~~~~~~~~~~~~~~~~~~~~\char40{}msum\char41{}\\
\tt ~import~Happstack\char46{}Server~~~~~~~~~~~~~~~~~~~\char40{}Response\char44{}~ServerPart\char44{}~\\
\tt ~~~~~~~~~~~~~~~~~~~~~~~~~~~~~~~~~~~~~~~~~~~Method\char40{}GET\char44{}~POST\char41{}\char44{}~methodM\\
\tt ~~~~~~~~~~~~~~~~~~~~~~~~~~~~~~~~~~~~~~~~~~~\char44{}~defaultBodyPolicy\char44{}~dir\char44{}~getDataFn\\
\tt ~~~~~~~~~~~~~~~~~~~~~~~~~~~~~~~~~~~~~~~~~~~\char44{}~look\char44{}~lookInput\char44{}~fileServe\char44{}~nullDir\\
\tt ~~~~~~~~~~~~~~~~~~~~~~~~~~~~~~~~~~~~~~~~~~~\char44{}~notFound\\
\tt ~~~~~~~~~~~~~~~~~~~~~~~~~~~~~~~~~~~~~~~~~~~\char44{}~nullConf\char44{}~ok\char44{}~simpleHTTP\char44{}~toResponse\\
\tt ~~~~~~~~~~~~~~~~~~~~~~~~~~~~~~~~~~~~~~~~~~~\char44{}~seeOther\char41{}\\
\tt ~import~Text\char46{}Blaze~~~~~~~~~~~~~~~~~~~~~~~~~as~B\\
\tt ~import~Text\char46{}Blaze\char46{}Html4\char46{}Strict~~~~~~~~~~~~as~B~hiding~\char40{}map\char41{}\\
\tt ~import~Text\char46{}Blaze\char46{}Html4\char46{}Strict\char46{}Attributes~as~B~hiding~\char40{}dir\char44{}~title\char41{}~\\
\tt ~\\
\tt ~import~PropChecker~as~T\\
\tt ~import~PropParser~as~P
\end{tabbing}

For now I will import my tautology check until I can further understand how
happstack works.

\begin{tabbing}\tt
~main~\char58{}\char58{}~IO~\char40{}\char41{}\\
\tt ~main~\char61{}~simpleHTTP~nullConf~\char36{}~propcheck
\end{tabbing}

For 'dir "feedback"' has methodM POST attached to it in order match on the
the specific HTTP request, and if it does match, then produce the page.
This way, anyone trying to make a request on 'feedback' will in
this case go back to the home page.  Alternatively, we can nest another msum in
'dir "feedback"' and generate an error message that refers to specifically to
this case. I have taken this approach because it allows me to decide if in the
future I decide to remove the 'errorPage' or add more possibilities simply by
adding or subtratcing elements from the nested msum.

nullDir will check if the path is non-empty, and if it is, it the handler will
move onto the next item, which is notExist, my personalized 404 message. I
prefer my own, because I believe it can add a sense of user-friendliness.

It's interesting to note, that since I am using overloaded strings to make life
easier using blazeHtml, the string "/" needs to have it's type declared
explicitly because it doesn't know which type to choose. Another way around this
is to move propCheck into separate module, in order to avoid having to declare
the types explicitly, and this will be preferred when the website becomes more
complex, in order to have cleaner code.

\begin{tabbing}\tt
~propcheck~\char58{}\char58{}~ServerPart~Response\\
\tt ~propcheck~\char61{}~\\
\tt ~~~~~msum~\char91{}~dir~\char34{}feedback\char34{}~\char36{}~msum~\char91{}~methodM~POST~\char62{}\char62{}~feedback\\
\tt ~~~~~~~~~~~~~~~~~~~~~~~~~~~~~~~~~~\char44{}~errorPage\\
\tt ~~~~~~~~~~~~~~~~~~~~~~~~~~~~~~~~~~\char93{}\\
\tt ~~~~~~~~~~\char44{}~dir~\char34{}static\char34{}~\char36{}~fileServe~\char91{}\char93{}~\char34{}\char46{}\char34{}~\\
\tt ~~~~~~~~~~\char44{}~nullDir~\char62{}\char62{}~propForm\\
\tt ~~~~~~~~~~\char44{}~notExist\\
\tt ~~~~~~~~~~\char93{}
\end{tabbing}

  --       , seeOther ("/" :: String) (toResponse ("/" :: String))

\begin{tabbing}\tt
~propForm~\char58{}\char58{}~ServerPart~Response\\
\tt ~propForm~\char61{}~ok~\char36{}~toResponse~\char36{}\\
\tt ~~~~~html~\char36{}~do\\
\tt ~~~~~~~B\char46{}head~\char36{}~do\\
\tt ~~~~~~~~~title~\char34{}Propositional~Equivalance~Checker\char34{}\\
\tt ~~~~~~~B\char46{}h1~\char36{}~do\\
\tt ~~~~~~~~~text~\char34{}Peter\char39{}s~Mega~Ultra~Propositional~Equivalence~Checker\char34{}\\
\tt ~~~~~~~B\char46{}body~\char36{}~do\\
\tt ~~~~~~~~~p~\char40{}string~\char36{}~question1\char41{}\\
\tt ~~~~~~~~~p~\char34{}Given\char58{}\char34{}\\
\tt ~~~~~~~~~ul~\char36{}~li~\char36{}~text~\char34{}R1L~is~not~equivalent~to~R1T\char34{}\\
\tt ~~~~~~~~~ul~\char36{}~li~\char36{}~text~\char34{}R2L~is~not~equivalent~to~R2T\char34{}\\
\tt ~~~~~~~B\char46{}div~\char36{}~do\\
\tt ~~~~~~~~~\\
\tt ~~~~~~~B\char46{}div~\char36{}~do\\
\tt ~~~~~~~~~p~\char40{}string~\char36{}~instructions\char41{}\\
\tt ~~~~~~~~~form~\char33{}~enctype~\char34{}multipart\char47{}form\char45{}data\char34{}~\char33{}~B\char46{}method~\char34{}POST\char34{}~\char33{}~action~\char34{}\char47{}feedback\char34{}~\char36{}~do\\
\tt ~~~~~~~~~~~~~~input~\char33{}~type\char95{}~\char34{}text\char34{}~\char33{}~name~\char34{}user\char95{}prop\char34{}~\char33{}~size~\char34{}40\char34{}~\char33{}~maxlength~\char34{}40\char34{}\\
\tt ~~~~~~~~~~~~~~input~\char33{}~type\char95{}~\char34{}submit\char34{}~\char33{}~name~\char34{}check\char95{}prop\char34{}~\char33{}~value~\char34{}Check~this~proposition\char34{}\\
\tt ~~~~~~~~~~~~~~input~\char33{}~type\char95{}~\char34{}reset\char34{}~\char33{}~name~\char34{}clear\char95{}prop\char34{}~\char33{}~value~\char34{}Clear\char34{}\\
\tt ~\\
\tt ~instructions~\char58{}\char58{}~String\\
\tt ~instructions~\char61{}~~\char34{}Enter~a~proposition~that~describes~this~situation\char46{}~\char34{}\\
\tt ~\\
\tt ~question1~\char58{}\char58{}~String\\
\tt ~question1~\char61{}~~\char34{}\char91{}\char46{}\char46{}\char46{}\char93{}~the~king~explained~to~the~prisoner~that~each~of~the~two~\char34{}\\
\tt ~~~~~~~~~~~\char43{}\char43{}~\char34{}rooms~contained~either~a~lady~or~a~tiger\char44{}~but~it~could~be~that~\char34{}\\
\tt ~~~~~~~~~~~\char43{}\char43{}~\char34{}there~were~tigers~in~both~rooms\char44{}~or~ladies~in~both~rooms\char44{}~or~\char34{}\\
\tt ~~~~~~~~~~~\char43{}\char43{}~\char34{}then~again\char44{}~maybe~one~room~contained~a~lady~and~the~other~room~\char34{}\\
\tt ~~~~~~~~~~~\char43{}\char43{}~\char34{}a~tiger\char46{}\char34{}~~~~~~~
\end{tabbing}

So far, I know how to get data from a form and display it, as well as do any
necessary calculations that gives me a value to display. My next problem is
getting the error message from the Parser in case the user enters a string
that can't be parsed, and displaying this error message to the user, which
should be helpful to the user.

Current Situation: 
\begin{enumerate}
        \item If a string that can't be parsed is encountered then 'feedback' does
           not load at all.
        \item I still need to organize the layout of my page and how responses for
           each case will be handled
\end{enumerate}

\begin{tabbing}\tt
~feedback~\char58{}\char58{}~ServerPart~Response\\
\tt ~feedback~\char61{}~\\
\tt ~~~~do~r~\char60{}\char45{}~getDataFn~\char40{}defaultBodyPolicy~\char34{}\char47{}tmp\char47{}\char34{}~1000~1000~1000\char41{}~\char36{}~look~\char34{}user\char95{}prop\char34{}\\
\tt ~~~~~~~ok~\char36{}~toResponse~\char36{}\\
\tt ~~~~~~~~~~html~\char36{}~do\\
\tt ~~~~~~~~~~~~B\char46{}head~\char36{}~do\\
\tt ~~~~~~~~~~~~~~title~\char34{}Prop~Feedback\char34{}\\
\tt ~~~~~~~~~~~~B\char46{}h2~\char36{}~do\\
\tt ~~~~~~~~~~~~~~text~\char34{}Feedback~on~given~Proposition\char34{}~~~~~~~~~~~\\
\tt ~~~~~~~~~~~~~~img~\char33{}~src~\char34{}\char47{}static\char47{}lambda\char46{}gif\char34{}~\char33{}~alt~\char34{}lambda\char34{}~\char33{}~width~\char34{}40\char34{}~\char33{}~height~\char34{}40\char34{}\\
\tt ~~~~~~~~~~~~B\char46{}body~\char36{}~do\\
\tt ~~~~~~~~~~~~~~mkBody~r\\
\tt ~~~~~where\\
\tt ~~~~~~~mkBody~\char40{}Left~errs\char41{}~\char61{}\\
\tt ~~~~~~~~~~~do~p~\char36{}~\char34{}The~following~error~occurred\char58{}\char34{}\\
\tt ~~~~~~~~~~~~~~mapM\char95{}~\char40{}p~\char46{}~string\char41{}~errs\\
\tt ~~~~~~~mkBody~\char40{}Right~theprop\char41{}~\char61{}~do\\
\tt ~~~~~~~~~~~~~~~~~B\char46{}h3~\char36{}~do\\
\tt ~~~~~~~~~~~~~~~~~~~text~\char34{}Analysis\char34{}\\
\tt ~~~~~~~~~~~~~~~~~isEquivalent~\char40{}P\char46{}evalProp~theprop~prop1~rests1\char41{}~theprop\\
\tt ~\\
\tt ~isEquivalent~~~~~~~~~~~~~~~\char58{}\char58{}~String~\char45{}\char62{}~String~\char45{}\char62{}~Html~b\\
\tt ~isEquivalent~\char34{}True\char34{}~~prop~~\char61{}~~p~\char40{}string~\char36{}~\char34{}Your~proposition~\char39{}\char34{}~\char43{}\char43{}~prop~\char43{}\char43{}~\\
\tt ~~~~~~~~~~~~~~~~~~~~~~~~~~~~~~~~~\char34{}\char39{}~correctly~describes~this~situation\char46{}\char34{}\char41{}\\
\tt ~isEquivalent~\char34{}False\char34{}~prop~~\char61{}~~do\\
\tt ~~~~~~~~~~~~~~~~~~~~~~~~~~~~~~~~~p~\char40{}string~\char36{}~\char34{}Your~proposition~\char39{}\char34{}~\char43{}\char43{}~prop~\char43{}\char43{}\\
\tt ~~~~~~~~~~~~~~~~~~~~~~~~~~~~~~~~~~\char34{}\char39{}~incorrectly~describes~this~situation\char46{}\char34{}\char41{}\\
\tt ~~~~~~~~~~~~~~~~~~~~~~~~~~~~~~~~~B\char46{}h4~\char36{}~text~\char34{}Here\char39{}s~a~tip\char58{}~\char34{}\\
\tt ~~~~~~~~~~~~~~~~~~~~~~~~~~~~~~~~~p~\char40{}string~\char36{}~\char34{}When~\char34{}~\char43{}\char43{}~evalDisagree~prop~prop1~rests1~\char43{}\char43{}\\
\tt ~~~~~~~~~~~~~~~~~~~~~~~~~~~~~~~~~~\char34{}Carefully~look~over~the~\char34{}~\char43{}\char43{}\\
\tt ~~~~~~~~~~~~~~~~~~~~~~~~~~~~~~~~~~\char34{}information~that~is~being~given~to~you~and~try~again\char46{}\char34{}\char41{}\\
\tt ~isEquivalent~error~~~prop~~\char61{}~~do\\
\tt ~~~~~~~~~~~~~~~~~~~~~~~~~~~~~~~~~p~\char40{}string~\char36{}~\char40{}\char34{}\char39{}\char34{}~\char43{}\char43{}~prop~\char43{}\char43{}~\char34{}\char39{}~is~not~a~correctly\char34{}~\char43{}\char43{}\\
\tt ~~~~~~~~~~~~~~~~~~~~~~~~~~~~~~~~~~~\char34{}~written~proposition\char46{}\char34{}\char41{}\char41{}\\
\tt ~~~~~~~~~~~~~~~~~~~~~~~~~~~~~~~~~B\char46{}h4~\char36{}~text~\char34{}Check~\char34{}\\
\tt ~~~~~~~~~~~~~~~~~~~~~~~~~~~~~~~~~p~\char40{}string~\char36{}~betterError~error\char41{}\\
\tt ~\\
\tt ~~\\
\tt ~\\
\tt ~\char45{}\char45{}~Enter~your~restrictions~here\\
\tt ~rests1~\char58{}\char58{}~Rests\\
\tt ~rests1~\char61{}~~\char91{}~Not~\char40{}Equiv~\char40{}Var~\char39{}a\char39{}\char41{}~\char40{}Var~\char39{}c\char39{}\char41{}\char41{}\\
\tt ~~~~~~~~~~~\char44{}~Not~\char40{}Equiv~\char40{}Var~\char39{}b\char39{}\char41{}~\char40{}Var~\char39{}d\char39{}\char41{}\char41{}\\
\tt ~~~~~~~~~~~\char93{}\\
\tt ~\\
\tt ~prop1~\char58{}\char58{}~String\\
\tt ~prop1~\char61{}~~\char34{}\char40{}a~v~b\char41{}~\char38{}~\char40{}c~v~d\char41{}\char34{}\\
\tt ~\\
\tt ~errorPage~\char58{}\char58{}~ServerPart~Response\\
\tt ~errorPage~\char61{}~ok~\char36{}~toResponse~\char36{}\\
\tt ~~~~~html~\char36{}~do\\
\tt ~~~~~~~B\char46{}head~\char36{}~do\\
\tt ~~~~~~~~~title~\char34{}Error~Page\char34{}\\
\tt ~~~~~~~B\char46{}h1~\char36{}~do\\
\tt ~~~~~~~~~text~\char34{}Oops\char33{}\char34{}\\
\tt ~~~~~~~B\char46{}body~\char36{}~do\\
\tt ~~~~~~~~~p~\char36{}~\char34{}It~seems~you~tried~to~check~your~feedback~without~submitting~a~proposition\char46{}\char34{}\\
\tt ~~~~~~~B\char46{}div~\char36{}~do\\
\tt ~~~~~~~~~p~\char36{}~\char34{}\char34{}\\
\tt ~\\
\tt ~notExist~\char58{}\char58{}~ServerPart~Response\\
\tt ~notExist~\char61{}~notFound~\char36{}~toResponse~\char36{}\\
\tt ~~~~~html~\char36{}~do\\
\tt ~~~~~~~B\char46{}head~\char36{}~do\\
\tt ~~~~~~~~~title~\char34{}Error~Page\char34{}\\
\tt ~~~~~~~B\char46{}h1~\char36{}~do\\
\tt ~~~~~~~~~text~\char34{}Sorry\char33{}\char34{}\\
\tt ~~~~~~~B\char46{}body~\char36{}~do\\
\tt ~~~~~~~~~p~\char36{}~\char34{}This~page~doesn\char39{}t~exist\char44{}~maybe~you~can~ask~for~it~to~be~in~the~next~version\char46{}\char34{}\\
\tt ~~~~~~~B\char46{}div~\char36{}~do\\
\tt ~~~~~~~~~p~\char36{}~\char34{}\char34{}\\
\tt ~\\
\tt ~\\
\tt ~\char45{}\char45{}~This~provides~a~better~error~message\\
\tt ~betterError~~~~~~~~~~~~~\char58{}\char58{}~String~\char45{}\char62{}~String\\
\tt ~betterError~err~~~~~~~~~\char61{}~~drop~8~\char40{}filter~\char40{}\char47{}\char61{}~\char39{}\char41{}\char39{}\char41{}~err\char41{}
\end{tabbing}
\end{document}
